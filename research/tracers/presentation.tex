\documentclass{beamer}

\usepackage{graphicx,subfigure,url}
\usepackage{pgfpages,multimedia}
\usepackage{pdfpc-commands}

% example themes
\usetheme{metropolis}
\metroset{numbering=fraction,progressbar=frametitle}
% \usecolortheme{seahorse}


% \setbeameroption{show notes}
% \setbeameroption{show notes on second screen=right}

\title{Monte-Carlo Tracer Particles}
\subtitle{RUM 2017, Nice}
\author{Corentin Cadiou}
\institute{IAP, CNRS}
\date{September 20, 2017}

\begin{document}

\frame[plain]{\titlepage}

\begin{frame}[plain]{Outline}
  \tableofcontents
\end{frame}

\section{Introduction}

\begin{frame}{Eulerian \& Lagrangian}
  Eulerian code (AMR like):
  \begin{itemize}
  \item no subgrid information
  \item no Lagrangian history of gas
  \end{itemize}

  {\em Is it possible to overcome this issue?}
\end{frame}
\begin{frame}{From passive scalar to tracers}

  \begin{itemize}
  \item<1-> Where does the gas go?
  \item<2-> Where does the gas come from?
  \item<3-> How much gas is recycled in stars? in AGNs?
  \item<4-> ... [TBC]
  \end{itemize}

\end{frame}

\begin{frame}{What's a tracer}
  What are the properties we want for tracers?
  \pause
  \begin{block}{Physical properties}
    \begin{itemize}
    \item Passive
    \item {\em Behave} like the gas on average
    \item Like individual $H/He$ nuclei
    \end{itemize}
  \end{block}

  \begin{block}{Computational properties}
    \begin{itemize}
    \item Cheap (CPU? RAM?)
    \item Go where the gas goes (star, sinks, grid, dust, ...)
    \end{itemize}
  \end{block}

\end{frame}

\section{Different methods}
\subsection{Velocity method}
\begin{frame}{Velocity method}

  Move tracers using tri-linear interpolation of the velocity
  \begin{equation}
    v^p_i = \text{interpolation}\sum_{\text{neighbor}\ j} v_j
  \end{equation}

  \begin{block}{Pros}
    \begin{itemize}
    \item
      smooth Lagrangian history (trace velocity)
    \item
      already implemented in RAMSES!
    \end{itemize}
  \end{block}
  \pause
  \begin{block}{Cons}
    \begin{itemize}
    \item
      \emph{does not} follow the gas density: $\int\text{d}S v^p_i\rho
      \neq \text{flux}$
    \item
      how to trace stars? AGN?
    \item
      quite CPU expensive
    \end{itemize}
  \end{block}

\end{frame}

\subsection{Monte Carlo method}
\begin{frame}{Monte Carlo method}

  \begin{columns}
    \column{0.6\textwidth}
    Move tracers following flux
    \begin{equation*}
      p_{i\rightarrow j} = \frac{\text{mass flux}_{i\rightarrow j}}{M_i}
    \end{equation*}

    \column{.35\textwidth}
    \vspace{.3em}
    \includegraphics[width=\columnwidth]{figures/tracer_particles_one_by_one}
  \end{columns}
% \end{frame}
% \begin{frame}{Monte Carlo method}
  \pause
  \begin{columns}
    \column{0.5\textwidth}
    \begin{block}{Pros}
      \begin{itemize}
      \item CPU cheap
      \item follow gas density
      \item precision $\propto N_\text{tracers}$
      \item move onto stars, sinks, \ldots{}
      \item \pause \textbf{now implemented!}
      \end{itemize}
    \end{block}

    \pause
    \column{0.5\textwidth}
    \begin{block}{Cons}
      \begin{itemize}
      \item RAM expensive
      \item noisy Lagrangian history
      \end{itemize}
    \end{block}
  \end{columns}
\end{frame}


\section{MC Implementation}

\begin{frame}{Scheme}
  \includegraphics[width=\columnwidth]{figures/tracer_particles}
\end{frame}

\begin{frame}{Equations}
  First and last equations:
  Let
  \[ M_{i,\text{out}} = \sum_{j\wedge i} M_{i\rightarrow j} \quad \text{ if $M_{i\rightarrow j} > 0$},\]
  then \textbf{for all tracer particles} in cell \(i\):
  \begin{align}
    p_{i, \text{out}} = \frac{M_{i,\text{out}}}{M_i}, &
    \quad\text{\# Proba. of going out of $i$} \\
    p_{i\rightarrow j} = \frac{M_{i\rightarrow j}}{M_{i,\text{out}}}, &
    \quad\text{\# Proba. of going from $i$ to $j$}
  \end{align}
  following \emph{S.Genel et al, 13}
\end{frame}

\begin{frame}{Algorithm}
  Algorithm for cell $i$ mass $M_i$, neighbors $j$:
  \begin{enumerate}
  \item<1-> Compute outgoing mass $M_{i,\text{out}}$ and $M_{i\rightarrow j}$.
  \item<2-> Compute outgoing proba $p_{i,\text{out}}=M_{i,\text{out}}/M_i$
    and $p_{i\rightarrow j} = M_{i\rightarrow j}/M_{i,\text{out}}$.
  \item<3-> For each particle:
    \begin{enumerate}
    \item Draw random number $r_j$.
    \item Select particles $r_j < p_{i,\text{out}}$.
    \end{enumerate}
  \item<4-> For each selected particle:
    \begin{enumerate}
    \item Draw random number $r_j'$.
    \item If $r_j' < p_{i\rightarrow j}$, move to cell $j$.
    \end{enumerate}
  \end{enumerate}
  \uncover<5>{\alert{Small flux limit: $N_\text{moved}\sim\text{Poisson distribution}(p)$}}
\end{frame}
\section{Is it working?}
% \subsubsection{Spoiler alert: yes}
\begin{frame}{Cosmo}
  \begin{columns}
    \column{.4\textwidth}
    \href{run:figures/MC_cosmo.avi?autostart&loop=false}{%
      \includegraphics[width=\columnwidth]{figures/MC_cosmo.png}}
    \column{.4\textwidth}
    \href{run:figures/density_cosmo.avi?autostart&loop=false}{%
      \includegraphics[width=\columnwidth]{figures/density_cosmo.png}}
    \column{.4\textwidth}
    \href{run:figures/vel_cosmo.avi?autostart&loop=false}{%
      \includegraphics[width=\columnwidth]{figures/vel_cosmo.png}}
  \end{columns}
\end{frame}

\fullFrameImage{figures/comparison/comparison_3.png}{\copyrightText{Cosmological
    Simulation, DM+hydro, left to right: MC tracer, gas and velocity tracers}}
\fullFrameImage{figures/info_00028_Projection_x_density_cropped.pdf}{\copyrightText{Cosmological
    Simulation, DM+hydro, left to right: MC tracer, gas and velocity tracers}}

% \begin{frame}
%   \frametitle{Power spectrum}
%   TODO: plot of the gas power spectrum (for a 256x256x256 grid)
% \end{frame}

\fullFrameMovie[loop=false]{figures/KH_movie.avi}{figures/info_00375_tracer_vs_gas_density}
% \begin{frame}{Kelvin-Helmoltz instability}
%   \href{run:figures/KH_movie.avi?autostart&loop=false}{\}
%   Here I want 3 movies in //
%   \href{figures/output.avi}{Movie}
% \end{frame}

\fullFrameMovie[loop=false]{figures/blob_movie.avi}{figures/info_00020_tracer_vs_gas_density.png}
% \begin{frame}{Blob test}
%   Blob test in slow-mow
% \end{frame}

\begin{frame}{Star formation}
  \begin{itemize}
  \item SF recipy: mechanical feedback
  \item homogeneous density
  \end{itemize}
  \centering
  \includegraphics[width=.8\columnwidth]{figures/star_formation}
\end{frame}

\begin{frame}
  \frametitle{Small galaxy}
  \begin{center} %
    \only<1>{Gas only

      \includegraphics[width=.8\textwidth]{figures/plots_galaxy/info_00030_Projection_x_density_ones.png}%
    }\only<2>{Gas and stars

      \includegraphics[width=.8\textwidth]{figures/plots_galaxy/info_00030_Projection_x_density_ones_star.png}%
    }\only<3>{Gas and star tracers

      \includegraphics[width=.8\textwidth]{figures/plots_galaxy/info_00030_Projection_x_density_ones_star_and_tracers.png}%
    }\only<4>{MC tracers

      \includegraphics[width=.8\textwidth]{figures/plots_galaxy/info_00030_Projection_x_MC_tracer_cic_ones.png}%
    }
  \end{center}
\end{frame}

\section{Discussion}
\begin{frame}
  \frametitle{Room for improvement}
  \begin{block}{TODO \& wishlist:}
    \begin{itemize}
    \item Get AGN feedback done (WIP).
    \item Other SN feedback.
    \item Quantify diffusion (esp. high flux limit)
    \item Explore other MC algorithms.
    \item \pause{} Get users!
    \end{itemize}
  \end{block}
  \uncover<2>{
    \begin{center}
      ``SAV'' at \alert{corentin.cadiou@iap.fr}
    \end{center}
  }
\end{frame}

% \section*{Advertisement}
% \begin{frame}{YT!}
%   yt now supports:
%   \begin{itemize}
%   \item sinks
%   \item RT
%   \item custom particle + fluid fields
%   \end{itemize}

%   \begin{center}
%     \alert{http://yt-project.com}
%   \end{center}

% \end{frame}


\begin{frame}[standout]
  Thank you!\\
  Questions?
\end{frame}


\section*{Advertisement}
\begin{frame}{YT!}
  yt now supports:
  \begin{itemize}
  \item sinks
  \item RT
  \item custom particle + fluid fields
  \item \alert{BSD} license (permissive)
  \end{itemize}

  \begin{center}
    \alert{http://yt-project.com}
  \end{center}

\end{frame}

\end{document}
