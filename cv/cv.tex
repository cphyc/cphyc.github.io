% -- Encoding UTF-8 without BOM
% -- XeLaTeX => PDF (BIBER)

\documentclass[]{friggeri-cv}
% \sethyphenation[]{french}{} % Add words between the {} to avoid them to be cut 

\begin{document}

\header{Corentin}{\ Cadiou}{Thèse - L'Origine de la Séquence de Hubble - IAP, CNRS}

%----------------------------------------------------------------------------------------
%	SIDEBAR SECTION
%----------------------------------------------------------------------------------------

\begin{aside} % In the aside, each new line forces a line break
\section{Contact}
72 avenue du Général Leclerc
75 014 Paris
France\vspace{0.6em}
+33 6 43 18 66 83\vspace{.6em}
\href{mailto:corentin.cadiou@iap.fr}{corentin.cadiou@iap.fr}
\href{http://cv.cphyc.me}{http://cv.cphyc.me}
\section{Langages}
Anglais -- avancé (114/120 au TOEFL)
Allemand -- intermédiaire
\section{Physique}
cosmologie, hydrodynamique, physique quantique, physique statistique, matière condensée, relativé restreinte et générale
\section{Informatique}
\textbf{— Avancé —}
\emph{MPI, OpenMP, Fortran, C}, C++, Python \& NumPy, OCaml, Linux, javascript, \vspace{.9em}
\textbf{— Intermédiaire —}
HTML, CSS, PHP, Bash, Maple, Assembler\vspace{.9em}
\textbf{— Simulations —}
RAMSES (MHD)
\end{aside}

%----------------------------------------------------------------------------------------
%	EDUCATION SECTION
%----------------------------------------------------------------------------------------
\section{Expérience professionnelle}
\begin{entrylist}
%------------------------------------------------
\entry
{2016--2019}
{Thèse}
{IAP, Paris, France}
{Origine de la séquence de Hubble: simulations numériques de l'accrétion de gaz froid sur les galaxies et modèles analytique de l'influence de l'environnement sur le biais d'assemblage des galaxies.}
%------------------------------------------------
\entry
{2016--2019}
{Monitorat}
{UPMC, Paris, France}
{Chargé de TD pour l'UE \emph{C}oncept et \emph{M}éthode de la \emph{P}hysique.}
%------------------------------------------------
\entry
{2015}
{Stage de 6 mois dans le privé}
{Linagora, Paris, France}
{Stage d'informatique : développement d'un éditeur de texte collaboratif basé sur un algorithme pair-à-pair..}
%-----------------------------------------------
\entry
{2014}
{Stage de recherche de 5 mois}
{UCSC, California, USA}
{Étude des géantes rouges : mise en évidence du rôle de la convection thermohaline et d'une instabilité magnétorotationnelle dans le transport du moment angulaire du noyau à l'enveloppe.}

\end{entrylist}

\section{Formation}

\begin{entrylist}
%------------------------------------------------
\entry
{2015--2016}
{M2 AAIS}
{ObsPM, UPMC, Diderot, Paris Sud \& ENS, Paris, France}
{Spécialité astrophysique : intérêt particulier pour les objets compacts, les grandes échelles et le calcul numérique. \emph{Classé 2/37, mention très bien}.}

%------------------------------------------------
\entry
{2015--2016}
{Diplôme de l'École Normale Supérieure (DENS)}
{ENS, Paris, France}
{Mineure en informatique :\@ algorithmique, structures et algorithmes aléatoires, 
informatique théorique, calculabilité.}

%------------------------------------------------
\entry
{2012--2014}
{L3 \& ICFP program}
{ENS, Paris, France}
{International Center for Fundamental Physics. \emph{Admis 19e} au concours d'entrée de l'ENS.}

%------------------------------------------------
\entry
{2010--2012}
{Classe préparatoire}
{Lycée Sainte-Geneviève, Versailles, France}
{Classé $\sim$ 5/35, équivalent d'une licence 2 \emph{mention très bien}.}

%------------------------------------------------
\end{entrylist}

%----------------------------------------------------------------------------------------
%	WORK EXPERIENCE SECTION
%----------------------------------------------------------------------------------------

\section{Expérience personnelle}

\begin{entrylist}
%------------------------------------------------
\entry
{2015}
{Google Hashcode}
{Paris, France}
{Classé 2e équipe au concours de programmation de Google ($\sim$130 équipes).}

%------------------------------------------------
\entry
{}
{Associatif}
{Paris, France}
{Vie étudiants de l'ENS, membre d’une ONG d’aide au développement au Bénin (construction d’une école, etc., 2006--).}

%------------------------------------------------
\end{entrylist}

%----------------------------------------------------------------------------------------
%	INTERESTS SECTION
%----------------------------------------------------------------------------------------

\section{Hobbies}
\textbf{Electronique et informatique} -- création d’un robot, de circuits électroniques, création d’un logiciel de gestion pour les associations de l’ENS, développement d’un logiciel pour l’académie de Strasbourg, …\\
\textbf{Voyages} -- Europe, Afrique, Amérique du Nord\\
\textbf{Sports} -- course à pied, basketball, badminton, …
\end{document}
